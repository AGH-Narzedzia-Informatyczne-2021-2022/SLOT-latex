\newpage
\section{Hubert Asztabski}
\label{sec:hubasz}

\setlength{\baselineskip}{0,5cm}

Tabela~\ref{tab:serialehubasz} przedstawia trzy seriale, które warto obejrzyć 

\setlength{\parskip}{0cm}
    \begin{tabular}{|c|c|c|p{5cm}|}
    \hline
        ID  &   Nazwa   &   Sezony  &   Krótki opis\\
        \hline
        1.  &   Breaking Bad    &   5   &   Gdy nauczyciel chemii dowiaduje się, że ma raka, postanawia rozpocząć produkcję metamfetaminy. \\
        \hline
        2.  &   The Walking Dead    &   11  &   Świat opanowały zombie. Grupka ocalałych szuka bezpiecznego schronienia. \\
        \hline
        3.  &   Mindhunter  &   2   &   Jest rok 1979. Dwóch agentów FBI zostaje wyznaczonych, aby przesłuchać osadzonych w więzieniu seryjnych morderców. \\
        \hline
    \end{tabular}
    \label{tab:serialehubasz}
    \begin{center}
    \caption{Tabela~\ref{tab:serialehubasz} Dobre seriale}
    \end{center}
    \par


Platformy na których można oglądnąc w/w seriale:
\setlength{\baselineskip}{0ex}
\setlength{\parskip}{0ex}
\begin{enumerate}
    \item Netflix
    \item Netflix, HBO GO, PLAY NOW, Canal+
    \item Netflix
\end{enumerate}

Cechy wspólne seriali (patrz Tabela~\ref{tab:serialehubasz}):
\begin{itemize}
    \item Wciągająca fabuła
    \item Dobrze napisane postacie
    \item Zachwycające zdjęcia
\end{itemize}
\setlength{\parskip}{1ex}
{\Large\textbf{Dlaczego \underline{Breaking Bad} jest najlepszy?}\par}
\textit{Breaking Bad} został nakręcony przez \textbf{Vince'a Gilligan'a}, który jest świetnym reżyserem. Znany jest z wspaniałych ujęć czy z płynnych przejśc między scenami, więc serial został nakręcony wzorowo. \\
Lecz to ta scena (Figure~\ref{fig:huell}) czyni \textit{Breaking Bad} najlepszym serialem na świecie.
\newpage

{\large \textit{\textbf{Szczupły mężczyzna leży na swoim ciężko zapracowanym i okupionym krwią, potem i łzami, dorobku, myśląc: "Czego by tu jeszcze dzisiaj nie zrobić?"}}}
\begin{figure}[htbp]
    \includegraphics[width=1.0\textwidth]{pictures/guy_in_paradise.png}
    \caption{Satysfakcja w pełni tego słowa znaczeniu}
    \label{fig:huell}
\end{figure}
\\
Przydatny wzór (do Figure~\ref{fig:huell}): $$d = \frac{m}{V} \biggl[\frac{kg}{m^3}\biggl]$$