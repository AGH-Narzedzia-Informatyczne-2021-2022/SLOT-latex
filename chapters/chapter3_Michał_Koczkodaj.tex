\section{Michał Koczkodaj}
\label{sec:Szczecin}

Zdjecie Szczecina (Figure~\ref{fig:Szczecin}).

\begin{figure}[htbp] % Co oznacza [htbp]? -> oznacza, że obraz pojawi się w odpowiednim miejscu w sekcji 
    \centering
    \includegraphics[width=0.5\textwidth]{pictures/Szczecin.jpg} % Jak sprawić, żeby obrazek był większy? -> trzeba zmienić wartość width
    \caption{Wały Chrobrego.}
    \label{fig:Szczecin}
\end{figure}

Tabela~\ref{tab:liczby} Ekstraklasa% Do czego służy \ref{}? -> odniesienie do tabeli w celu prowadzenia numeracji

\begin{table}[htbp]
%Michał Koczkodaj
\centering
\begin{tabular}{||c c c ||} \hline
drużyna        &   liczba meczy& punkty \\ \hline
Lech Poznań       & 12     & 27 pkt \\ \hline
Raków Częstochowa & 11     & 24 pkt \\ \hline
Lechia Gdańsk     & 12     & 23 pkt \\ \hline
Pogoń Szczecin    & 12     & 22 pkt \\ \hline
\end{tabular}
\label{tab:liczby}
\caption{Ekstraklasa tabela stan na 29.10}
\end{table}


% Dlaczego wyrażenie przeskoczyło do nowej linijki? -> Ponieważ zostało użyte \[\]
Here is also a worth seeing equation: \[E=mc^2\]

Mój przykład (prawo Pascala) :
\begin{eqnarray*}
p_2-p_1=-p*g*(h_2-h_1)
\end{eqnarray*}
Moja lista:
\begin{itemize}
  \item Warszawa to stolica Polski.
  \item Szczecin to trzecie najwieksze miasto w Polsce pod wzgledem powierzchi.
  \item Krakow jest super.
\end{itemize}
Moja lista numerowana:
\begin{enumerate}
  \item Warszawa - 1,8 mln
  \item Kraków - 0,8 mln
  \item Łódź - 0,7 mln
\end{enumerate}

\setlength{\parindent}{10ex}
\textbf{Szczecin} Zdjęcie (\ref{fig:Szczecin}) - stolica województwa Zachodniopomorskiego. Miasto położone na pobrzeżu Szczecińskim nad Odrą i jeziorem Dąbie.
\par
Według danych z GUS miasto liczy około \underline{400} tys mieszkańców. Pogoń Szczecin to jedna z najmocniejszych drużyn Ekstraklasy i zajmuje obecnie czwarte miejsce tabela
(\ref{tab:liczby})
\par
\setlength{\parindent}{0ex}

Moja lista nienumerowana ze zmianą kropek na F:

\begin{itemize}
  \item[F] rancja jest w Europie.
  \item[F] lorencja to bardzo ładne miasto.
  \item[F] loryda jest słoneczna.
\end{itemize}

