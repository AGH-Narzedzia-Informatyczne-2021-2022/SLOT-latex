\section{Bartłomiej Stępniewski}
\label{sec:bstepniewski}
\paragraph{Pierwsze 3 kraje z największym PKB(2020):}
\begin {enumerate}
\item USA
\item Chiny
\item Japonia
\end {enumerate}

\paragraph{Inne kraje z relatywnie dużym PKB(2020):}
\begin {itemize}
\item Niemcy
\item Francja
\item Wielka Brytania
\item Włochy
\end {itemize}

Ale nawet USA czy Chiny nie mają znaczenia w skali całej galaktyki... (Patrz Figure~\ref{fig:galatyka}).

\begin{figure}[htbp]
    \centering
    \includegraphics[width=0.7\textwidth]{pictures/galaktyka.jpg}
    \caption{Droga mleczna}
    \label{fig:galatyka}
\end{figure}

Prawdodobieństwo alternatywy dwóch zdarzeń:
\[P(A \cup B) = P(A) + P(B) - P(A \cap B)\]
\newpage
\paragraph{Tabela:}
Na wielkość PKB niewątpliwie ma wpływ jakość systemu edukacji w danym kraju. Tu dla przykładu, najczęstsza skala ocen w polskich szkołach(Patrz Table~\ref{tab:oceny}):
\begin{table}[htdp]
\centering
\begin{tabular}{|l|l|l|} 
\hline
\textbf{Ocena słownie: } & \textbf{Ocena liczbowo: } & \textbf{Wynik procentowy: }  \\ 
\hline
Niedostateczny           & 1                         & 50\%                         \\ 
\hline
Mierny                   & 2                         & 51\%-60\%                    \\ 
\hline
Dostateczny              & 3                         & 61\%-75\%                    \\ 
\hline
Dobry                    & 4                         & 76\%-90\%                    \\ 
\hline
Bardzo dobry             & 5                         & 91\%-100\%                   \\ 
\hline
Celujący                 & 6                         & 100\%(zadania dodatkowe      \\
\hline

\end{tabular}
\label{tab:oceny}
\caption{Skala ocen w Polsce}
\end{table}

\paragraph{}
{\centering\textbf{Inwokacja - Pan Tadeusz}}
\newline
\newline
Litwo, ojczyzno moja!\\
Ty jesteś jak zdrowie -\\
Ile cię trzeba cenić...

\paragraph{Nowe rozdanie w geopolityce globalnej:}
\underline{Czy Chiny zastąpią USA w roli hegemona??}
\\

Pierwszy raz \underline{od przeszło 100 lat} hegemonia USA jest zagrożona - fundamentalnie, od strony gospodarczej - eksperci przewidują bowiem, że PKB Chin w ciągu maksymalnie 15-20 lat przewyższy PKB amerykańskie.

